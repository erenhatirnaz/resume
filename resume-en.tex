%!TEX TS-program = xelatex
%!TEX encoding = UTF-8 Unicode
% Awesome CV LaTeX Template for CV/Resume
%
% This template has been downloaded from:
% https://github.com/posquit0/Awesome-CV
%
% Original author:
% Claud D. Park <posquit0.bj@gmail.com>
% http://www.posquit0.com
%
% Modifications by:
% Junhao Dong <junhao.dong96@gmail.com>
%
% Template license:
% CC BY-SA 4.0 (https://creativecommons.org/licenses/by-sa/4.0/)
%

%-------------------------------------------------------------------------------
% CONFIGURATIONS
%-------------------------------------------------------------------------------
% A4 paper size by default, use 'letterpaper' for US letter
\documentclass[11pt, a4paper]{awesome-cv}

% Configure page margins with geometry
\geometry{left=1.4cm, top=.8cm, right=1.4cm, bottom=1.8cm, footskip=.5cm}

% Specify the location of the included fonts
\fontdir[fonts/]

% Color for highlights
% Awesome Colors: awesome-emerald, awesome-skyblue, awesome-red, awesome-pink, awesome-orange
%                 awesome-nephritis, awesome-concrete, awesome-darknight
\colorlet{awesome}{awesome-orange}
% Uncomment if you would like to specify your own color
%% \definecolor{awesome}{HTML}{FE8091}

% Colors for text
% Uncomment if you would like to specify your own color
% \definecolor{darktext}{HTML}{414141}
% \definecolor{text}{HTML}{333333}
% \definecolor{graytext}{HTML}{5D5D5D}
% \definecolor{lighttext}{HTML}{999999}

% Set false if you don't want to highlight section with awesome color
\setbool{acvSectionColorHighlight}{true}

\usepackage{hyperref}
\hypersetup{pdfauthor={Eren Hatırnaz},%
            pdftitle={Eren Hatırnaz - Digital Product Owner & Back-End Developer},%
            pdfsubject={Resume},
            pdfkeywords={resume;product owner;scrum;agile;cv;back-end;php;laravel;linux;bash;aws}
}

% If you would like to change the social information separator from a pipe (|) to something else
\renewcommand{\acvHeaderSocialSep}{\quad\bullet\quad}

\makeatletter
\patchcmd{\@sectioncolor}{\color}{\mdseries\color}{}{}
\makeatother

%-------------------------------------------------------------------------------
%	PERSONAL INFORMATION
%	Comment any of the lines below if they are not required
%-------------------------------------------------------------------------------
% Available options: circle|rectangle,edge/noedge,left/right
% \photo[rectangle,edge,right]{profile}
\name{Eren}{Hatırnaz}
\position{Digital Product Owner{\enskip\cdotp\enskip}Back-End Developer{\enskip\cdotp\enskip}Computer Engineer}
%% \position{Back-End Developer{\enskip\cdotp\enskip}Bilgisayar Mühendisi}
\address{Rize, Turkey}
%% \address{Rize, Türkiye}

\mobile{(530) 967-8600}
\email{erenhatirnaz@gmail.com}
\dateofbirth{March 18st, 1995}
%% \dateofbirth{18 Mart 1995}
\homepage{erenhatirnaz.com}
\github{erenhatirnaz}
\linkedin{erenhatirnaz}
% \gitlab{gitlab-id}
% \stackoverflow{1691394}{eren-hatirnaz}
\twitter{@erenhatirnaz}
% \skype{skype-id}
% \reddit{reddit-id}
%% \extrainfo{}

%-------------------------------------------------------------------------------
\begin{document}

% Print the header with above personal informations
% Give optional argument to change alignment(C: center, L: left, R: right)
\makecvheader[C]

% Print the footer with 3 arguments(<left>, <center>, <right>)
% Leave any of these blank if they are not needed
\makecvfooter
  {\today}
  {Eren Hatırnaz~~~·~~~Résumé}
  {\thepage}

%% \makecvfooter
%%   {\today}
%%   {Eren Hatırnaz~~~·~~~Özgemiş}
%%   {\thepage}

%-------------------------------------------------------------------------------
%	CV/RESUME CONTENT
%	Each section is imported separately, open each file in turn to modify content
%-------------------------------------------------------------------------------
\cvsection{Summary}

\begin{cvparagraph}
I am a software developer who always keeps simplicity in mind while writing code,
and pays attention to learning new things and using what I have learned in
appropriate places. I am a fast learner and easily adaptable. I continue to
improve myself in problem solving and building algorithms rather than a specific
programming language. I like to collect feedback on my work.

I am also a supporter of Open Source and Free Software. I like to contribute to
the open source projects I use and develop open source projects/libraries.
\end{cvparagraph}

\cvsection{Education}

\begin{cventries}
  \cventry
    {Computer Engineering, GPA: 3.34/4.0} % Degree
    {Near East University} % Institution
    {Nicosia, North-Cyprus} % Location
    {June 2018} % Date(s)
    {
      \begin{cvitems} % Description(s) bullet points
         \item {\textbf{Rize University}, GPA: 3.70/4.0, Rize, Turkey --- \entrydatestyle{2013 - 2015}}
         \vspace{0.5mm}
         \item {\textbf{Sehit Cavit Koroglu Anatolian High School}, GPA: 60.08/100, Rize, Turkey --- \entrydatestyle{2009 - 2013}}
      \end{cvitems}
    }
\end{cventries}

\cvsection{Experience}

\begin{cventries}
  \cventry
    {Mobillium} % Organization
    {Digital Product Owner} % Job title
    {Remote} % Location
    {Ocak 2023 - PRESENT} % Date(s)
    {
      \begin{cvitems} % Description(s) bullet points
        \item Su iste: Water delivery mobile app. {\href{https://suiste.com/en}{suiste.com}}
      \end{cvitems}
    }
  \cventry
    {Mobillium} % Organization
    {Mid. Level Back-End Developer} % Job title
    {Remote} % Location
    {June 2022 - PRESENT} % Date(s)
    {
      \begin{cvitems} % Description(s) bullet points
        \item AWS, Queues, Rest API, CI/CD
      	\item Jr. Back-End Developer, March 2021 - June 2022
      \end{cvitems}
    }
\end{cventries}

\cvsection{Skills}

\begin{cvskills}
  \cvskill
    {Languages} % Type
    {PHP, Bash, JavaScript, Ruby, Python} % Skillset

  \cvskill
    {Cloud} % Type
    {AWS (EC2, RDS, S3, Cloudformation, CloudFront, CloudWatch, Route 53)} % Skillset

  \cvskill
    {Frameworks} % Type
    {Laravel, VueJS} % Skillset

  \cvskill
    {Tools} % Type
    {Git, GNU/Linux, Docker} % Skillset
\end{cvskills}

\cvsection{Projects}

\begin{cventries}
  \cventry
    {Ruby} % Empty position
    {\href{https://github.com/erenhatirnaz/angelco-startup-parser}{angelco-startup-parser}} % Project
    {} % Empty location
    {} % Empty date
    {
      \begin{cvitems} % Description(s) bullet points
        \item {A project that contains two cli tool for fetchting startup datas from ancel.co and converts to csv files}
		    \item {With the output of these tools, generated this
        graph: \href{https://graphcommons.com/graphs/8da5327d-7829-4dfe-b60b-4c0bda956b2a}{Discover
        IoT Initiatives} - IoT Startups which have at least 100 followers on
        angel.co and their markets in 2016.}
        \item {The graph is presented in \href{http://www.icits.org/}{ICITS 2016} at Tokyo, Japan.}
      \end{cvitems}
    }

  \cventry
    {Bash} % Empty position
    {\href{https://github.com/erenhatirnaz/mgm-radar}{mgm-radar}} % Project
    {} % Empty location
    {} % Empty date
    {
      \begin{cvitems} % Description(s) bullet points
      	\item {A command line tool that downloads meteorological radar images
      	from mgm.gov.tr (Turkey's national meteorology agency)}
      \end{cvitems}
    }
\end{cventries}

\cvsection{Publications}

\begin{cventries}
  \cventry
    {DOI: 10.1007/s11042-020-08659-2} % Empty position
    {\href{https://doi.org/10.1007/s11042-020-08659-2}{A novel framework and concept-based semantic search Interface for abnormal crowd behaviour analysis in surveillance videos}} % Project
    {Multimedia Tools and Applications, Springer} % Empty location
    {January 2020} % Empty date
    {
      \begin{cvitems} % Description(s) bullet points
        \item {A single-page web application that provides detailed video search, written in VueJS.}
		    \item {Conducted user experiments to show the difference between keyword-based search methods and semantic video search}
        \item {Other major technologies I use in this project: Apache Jena Fuseki, RDF, SPARQL, Docker, NodeJS}
        \item {The source code is available in the GitHub repo: \href{https://github.com/erenhatirnaz/semantic-video-search}{erennhatirnaz/semantic-video-search}}
      \end{cvitems}
    }

\end{cventries}

%% \cvsection{Hakkımda}

\begin{cvparagraph}
Kod yazarken sadeliği hep aklında bulunduran, yeni şeyler öğrenmeyi ve
öğrendiklerini uygun yerlerde kullanmaya dikkat eden bir yazılım geliştiriciyim.
Hızlı öğrenen ve kolay adapte olabilen bir yapıya sahibim. Bir programlama
dilinden ziyade kendimi problem çözme ve algoritma kurma konusunda geliştirmeye
devam ediyorum. Yaptığım işler hakkında geri bildirim toplamayı severim.

Ayrıca Açık Kaynak ve Özgür Yazılım destekçisiyim. Kullandğım açık kaynak
projelere katkıda bulunmayı ve açık kaynak projeler/kütüphaneler geliştirmeyi
severim.
\end{cvparagraph}

%% \cvsection{Eğitim}

\begin{cventries}
  \cventry
    {Bilgisayar Mühendisliği, Diploma Notu: 3.34/4.0} % Degree
    {Yakın Doğu Üniversitesi} % Institution
    {Lefkoşa, KKTC} % Location
    {Haziran 2018} % Date(s)
    {
      \begin{cvitems} % Description(s) bullet points
         \item {\textbf{Rize Üniversitesi}, Diploma Notu: 3.70/4.0, Rize, Türkiye --- \entrydatestyle{2013 - 2015}}
         \vspace{0.5mm}
         \item {\textbf{Şehit Cavit Köroğlu Anadolu Lisesi}, Diploma Notu: 60.08/100, Rize, Türkiye --- \entrydatestyle{2009 - 2013}}
      \end{cvitems}
    }
\end{cventries}

%% \cvsection{Deneyim}

\begin{cventries}
  \cventry
    {Mobillium} % Organization
    {Digital Product Owner} % Job title
    {Uzaktan} % Location
    {Ocak 2023 - Devam Ediyor} % Date(s)
    {
      \begin{cvitems} % Description(s) bullet points
        \item Su iste: Su siparişi mobil uygulaması. {\href{https://suiste.com}{suiste.com}}
      \end{cvitems}
    }
  \cventry
    {Mobillium} % Organization
    {Mid. Level Back-End Developer} % Job title
    {Uzaktan} % Location
    {Haziran 2022 - Devam Ediyor} % Date(s)
    {
      \begin{cvitems} % Description(s) bullet points
        \item AWS, Queues, Rest API, CI/CD
      	\item Jr. Back-End Developer, Mart 2021 - Haziran 2022
      \end{cvitems}
    }

\end{cventries}

%% \cvsection{Yetenekler}

\begin{cvskills}
  \cvskill
    {Programlama Dilleri} % Type
    {PHP, Bash, JavaScript, Ruby, Python} % Skillset

  \cvskill
    {Cloud} % Type
    {AWS (EC2, RDS, S3, Cloudformation, CloudFront, CloudWatch, Route 53)} % Skillset

  \cvskill
    {Framework'ler} % Type
    {Laravel, VueJS} % Skillset

  \cvskill
    {Araçlar} % Type
    {Git, GNU/Linux, Docker} % Skillset
\end{cvskills}

%% \cvsection{Projeler}

\begin{cventries}
  \cventry
    {Ruby} % Empty position
    {\href{https://github.com/erenhatirnaz/angelco-startup-parser}{angelco-startup-parser}} % Project
    {} % Empty location
    {} % Empty date
    {
      \begin{cvitems} % Description(s) bullet points
        \item {Ancel.co sitesinden startup bilgilerini çeken ve bunları CSV
        dosyalarına dönüştüren iki komut satırı aracı içeren bir projedir.}
		    \item {Bu projenin çıktılarıyla şu grafik
        oluşturulmuştur: \href{https://graphcommons.com/graphs/8da5327d-7829-4dfe-b60b-4c0bda956b2a}{Discover
        IoT Initiatives} - IoT Startups which have at least 100 followers on
        angel.co and their markets in 2016..}
        \item Oluşturulan grafik ise akademik bir konferansda
        (\href{http://www.icits.org/}{ICITS 2016} at Tokyo, Japan) sunulmuştur.
\end{cvitems}
    }

  \cventry
    {Bash} % Empty position
    {\href{https://github.com/erenhatirnaz/mgm-radar}{mgm-radar}} % Project
    {} % Empty location
    {} % Empty date
    {
      \begin{cvitems} % Description(s) bullet points
      	\item {mgm.gov.tr üzerinden meteorolojik radar görüntüsü indirme aracıdır.}
      \end{cvitems}
    }
\end{cventries}

%% \cvsection{Yayınlar}

\begin{cventries}
  \cventry
    {DOI: 10.1007/s11042-020-08659-2} % Empty position
    {\href{https://doi.org/10.1007/s11042-020-08659-2}{A novel framework and concept-based semantic search Interface for abnormal crowd behaviour analysis in surveillance videos}} % Project
    {Multimedia Tools and Applications, Springer} % Empty location
    {January 2020} % Empty date
    {
      \begin{cvitems} % Description(s) bullet points
        \item {Bu çalışma kapsamında videolar üzerinde detaylı arama yapmaya olanak sağlayan bir web arayüzü geliştirdim.}
		    \item {Semantic video araması ve keyword-based arama arasındaki farkı gösterebilmek için kullanıcı deneyleri yaptım.}
        \item {Proje genelinde kullandığım teknolojiler: VueJS, Apache Jena Fuseki, RDF, SPARQL, Docker, NodeJS}
        \item {Projenin kaynak kodları GitHub reposunda mevcuttur: \href{https://github.com/erenhatirnaz/semantic-video-search}{erennhatirnaz/semantic-video-search}}
      \end{cvitems}
    }

\end{cventries}


%-------------------------------------------------------------------------------
\end{document}
